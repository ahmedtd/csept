\documentclass[titlepage]{article}

\usepackage[paper=letterpaper, margin=2.5cm]{geometry}
\usepackage{parskip}
\usepackage{listings}
\usepackage{microtype}
\usepackage[htt]{hyphenat}
\usepackage{xspace}

\newcommand{\csept}{\texttt{csept}\space}
\newcommand{\gcov}{\texttt{gcov}\space}
\newcommand{\cmake}{\texttt{CMake}\space}
\newcommand{\ctest}{\texttt{CTest}\xspace}

\title{%
CSCE 431: Assignment 6 \\
\csept Calendaring System \\
Build System Overview
}

\author{%
Taahir Ahmed \\
Rachel Flores-Meath \\
Andrew Funderburgh \\
Maggie O'Brien \\
Patrick Smith
}

\date{%
19 April 2012
}

\begin{document}
\maketitle
\tableofcontents

\section{Test System Overview}

The goal of the \csept calendaring system is to provide a completely
cross-platform calendar solution for professionals and students, promoting
interoperability by supporting many input and output methods and formats for
calendaring data.  This ambitious goal necessitates a flexible build and testing
system, as well as widespread use of cross-platform solutions in order to
minimize code duplication.  To this end, the \csept project has adopted a build
and testing system using \cmake and \ctest, integrated with the Boost unit
testing framework.

\subsection{CMake}

\subsection{CTest}

\subsection{Boost Unit Test Framework}

\subsection{Code Coverage Testing (\gcov)}

\section{Test Registration}

\subsection{Unit Tests}

\subsection{Pre-, Post-, and Invariant-Tests}

\section{Test Execution}

\end{document}